\let\negmedspace\undefined
\let\negthickspace\undefined
\documentclass[journal,12pt,twocolumn]{IEEEtran}
\usepackage{cite}
\usepackage{amsmath,amssymb,amsfonts,amsthm}
\usepackage{algorithmic}
\usepackage{graphicx}
\usepackage{textcomp}
\usepackage{xcolor}
\usepackage{txfonts}
\usepackage{listings}
\usepackage{enumitem}
\usepackage{mathtools}
\usepackage{gensymb}
\usepackage{comment}
\usepackage[breaklinks=true]{hyperref}
\usepackage{tkz-euclide} 
\usepackage{listings}
\usepackage{gvv}                                        
%\def\inputGnumericTable{}                                 
\usepackage[latin1]{inputenc}                                
\usepackage{color}                                            
\usepackage{array}                                            
\usepackage{longtable}                                       
\usepackage{calc}                                             
\usepackage{multirow}                                         
\usepackage{hhline}                                           
\usepackage{ifthen}                                           
\usepackage{lscape}
\usepackage{tabularx}
\usepackage{array}
\usepackage{float}


\newtheorem{theorem}{Theorem}[section]
\newtheorem{problem}{Problem}
\newtheorem{proposition}{Proposition}[section]
\newtheorem{lemma}{Lemma}[section]
\newtheorem{corollary}[theorem]{Corollary}
\newtheorem{example}{Example}[section]
\newtheorem{definition}[problem]{Definition}
\newcommand{\BEQA}{\begin{eqnarray}}
\newcommand{\EEQA}{\end{eqnarray}}
\newcommand{\define}{\stackrel{\triangle}{=}}
\theoremstyle{remark}
\newtheorem{rem}{Remark}

% Marks the beginning of the document
\begin{document}
\bibliographystyle{IEEEtran}
\vspace{3cm}

\title{Chapter 2 \\ Complex Numbers}
\author{AI24BTECH11029-Rudrax Garwa}
\maketitle
\newpage
\bigskip

\renewcom mand{\thefigure}{\theenumi}
\renewcommand{\thetable}{\theenumi}
\begin{ducument}
\item 14. If $z^2+z+1=0$ ,where z is an imaginary number , then the value of $\left(z+\frac{1}{z}\right)^2 + \left(z^2+\frac{1}{z^2}\right)^2 + \left(z^3 +\frac{1}{z^3}\right)^2 +....+\left(z^6 +\frac{1}{z^6}\right)^2$ is 
\hfill{\color{magenta}[2006]}
\begin{multicols}{4}
\begin{enumerate}
\item 18
\item 54
\item 6
\item 12
\end{enumerate}
\end{multicols}
\item 15. If $|z+4|\leq 3$ , then the maximum value of $|z+1|$ is
\hfill{\color{magenta}[2007]}
\begin{multicols}{4}
\begin{enumerate}
\item 6
\item 0
\item 4
\item 10
\end{enumerate}
\end{multicols}
\item 16. The conjugate of a complex is $\frac{1}{i-1}$ then that complex number is
\hfill{\color{magenta}[2008]}
\begin{multicols}{1}
\begin{enumerate}
\item $\frac{-1}{i-1}$
\item $\frac{1}{i+1}$
\item $\frac{-1}{i+1}$
\item $\frac{1}{i-1}$
\end{enumerate}
\end{multicols}
\item 17. Let R be the real line.Consider the following subsets of the plane $R×R$:
S=\{(x,y):y=x+1 and 0<x<2\}
T=\{(x,y):x-y is an integer\},
Which one of the following is true ?
\hfill{\color{magenta}[2008]}
\begin{multicols}{1}
\begin{enumerate}
\item Neither S nor T is an equivalence relation on R
\item Both S and T are equivalence relation on R
\item S is an equivalence relation on R but T is not
\item T is an equivalence relation on R but S is not
\end{enumerate}
\end{multicols}
\item 18. The number of complex numbers z such that $|z-1|=|z+1|=|z-i|$ equals
\hfill{\color{magenta} [2010]}
\begin{multicols}{4}
\begin{enumerate}
\item 1
\item 2
\item $\infty$
\item 0
\end{enumerate}
\end{multicols}
\item 19. Let $\alpha,\beta$ be real and z be a complex number. If $z^2 +z\alpha +\beta =0$  has two distinct roots on the line $Rez=1$,then it is necessary that:
\hfill{\color{magenta}[2011]}
\begin{multicols}{2}
\begin{enumerate}
\item $\beta \in (-1,0)$
\item $|\beta|=1$
\item $\beta\in (1,\infty)$
\item $\beta\in (0,1)$
\end{enumerate}
\end{multicols}
\item 20. If $\omega(\neq1)$ is the cube root of unity, and $\left(1+\omega \right)^7= A+ B\omega$ .Then $(A,B)$ equals
\hfill{\color{magenta}[2011]}
\begin{multicols}{2}
\begin{enumerate}
\item $(1,1)$
\item $(1,0)$
\item $(-1,1)$
\item $(0,1)$
\end{enumerate}
\end{multicols}	
\end{enumerate}
\end{multicols}
\item 21. If $z \neq 1 and \frac{z^2}{z-1}$ is real,then the point represented by the complex number z lies :
\hfill{\color{magenta}[2012]}
\begin{multicols}{2}
\begin{enumerate}
\item either on the real axis or on a circle not passing through the origin
\item on a circle with centre at the origin
\item either on the real axis or on a circle not passing through the origin 
\item on imaginary axis
\end{enumerate}
\end{multicols}
\item 22. If z is a complex number of unit modulus and arguement $\theta$,then the $arg\left(\frac{1+z}{1+conjugate of z}\right)$ equals:
\hfill{\color{magenta}[JEE M 2013]}
\begin{multicols}{4}
\begin{enumerate}
\item $-\theta$
\item $\frac{\pi}{2}-\theta$
\item $\theta$
\item $\pi-\theta$
\end{enumerate}
\end{multicols}
\item 23. If z is a complex number such that $|z|\geq 2$,then the minimum value of $|z+\frac{1}{2}|$ :
\hfill{\color{magenta}[JEE M 2014]}
\begin{multicols}{2}
\begin{enumerate}
\item is strictly greater than $\frac{5}{2}$
\item is strictly greater than $\frac{3}{2}$ but less than $\frac{5}{2}$
\item is equal to $\frac{5}{2}$
\item lies in the interval $(1,2)$
\end{enumerate}
\end{multicols}
\item 24. A complex number z is said to be unimodular if $|z|=1$. Suppose $z_1 and z_2$ are complex numbers such that $\frac{z_1 -2z_2}{2-z_1conjugate ofz_2}$ is unimodular and $z_2$ is not unimodular. Then the point $z_1$ lies on a:
\hfill{\color{magenta}[JEE M 2016]}
\begin{multicols}{2}
\begin{enumerate}
\item circle of radius 2
\item circle of radius $\surd2$
\item straight line parallel to x-axis
\item straight line parallel to y-axis
\end{enumerate}
\end{multicols}
\item 25.A value of $\theta$ for which $\frac{2+3isin\theta}{1-2isin\theta}$ is purely imaginary is :
\hfill{\color{magenta}[JEE M2016]}
\begin{multicols}{2}
\begin{enumerate}
\item $sin^{-1}\left(\frac{\surd3}{4}\right)$
\item $sin^{-1}\left(\frac{1}{\surd3}\right)$
\item $\frac{\pi}{3}$
\item $\frac{\pi}{6}$
\end{enumerate}
\end{multicols}
\item 26.Let  $A=\{\theta \in \left(\frac{\pi}{2},\pi\right) :\frac{3+2isin\theta}{1-2isin\theta}$is purely imaginary.\} Then the sum of element in A is:
\hfill{\color{magenta}[JEE M 2019-9 Jan (M)]}
\begin{multicols}{2}
\begin{enumerate}
\item $\frac{5\pi}{6}$
\item $\pi$
\item $\frac{3\pi}{4}$
\item $\frac{2\pi}{3}$
\end{enumerate}
\end{multicols}
\item 27.let $\alpha$ and $\beta$ be two roots of the equation $x^2+2x+2=0$,then $\alpha^{15}+\beta^{15}$ is equal to :
\hfill{\color{magenta}[JEE M 2019-Jan (M)]}
\begin{multicols}{2}
\begin{enumerate}
\item $-256$
\item $512$
\item $-512$
\item $256$
\end{enumerate}
\end{multicols}
\item 28. All the points in the set 
   $S=\left\{\frac{\alpha+i}{\alpha-i}:\alpha \in R\right\}$ $(i=\surd{-1})$ lie on a:
\hfill{\color{magenta}[JEE M 2019-9 April (M)]}
\begin{multicols}{2}
\begin{enumerate}
\item straight line whose slope is 1
\item circle whose radius is 1
\item circle whose radius is $\surd2$
\item straight line whose slope is -1
\end{enumerate}
\end{multicols}
\end{enumerate}
\end{document}
