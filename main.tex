\let\negmedspace\undefined
\let\negthickspace\undefined
\documentclass[journal,12pt,twocolumn,article]{IEEEtran}
\usepackage{cite}
\usepackage{color,soul}
\usepackage{amsmath,amssymb,amsfonts,amsthm}
\usepackage{algorithmic}
\usepackage{graphicx}
\usepackage{textcomp}
\usepackage{xcolor}
\usepackage{txfonts}
\usepackage{listings}
\usepackage{enumitem}
\usepackage{mathtools}
\usepackage{gensymb}
\usepackage{comment}
\usepackage[breaklinks=true]{hyperref}
\usepackage{tkz-euclide}
\usepackage{listings}
\usepackage{multicol}
\usepackage{gvv}
\usepackage[dvipsnames]{xcolor}
\def\inputGnumericTable{}                           
\usepackage[latin1]{inputenc}                       
\usepackage{color}
\usepackage{array}                                  
\usepackage{longtable}                              
\usepackage{calc}                                   
\usepackage{multirow}
\usepackage{hhline}
\usepackage{ifthen}
\usepackage{lscape}
\newtheorem{theorem}{Theorem}[section]
\newtheorem{problem}{Problem}
\newtheorem{proposition}{Proposition}[section]
\newtheorem{lemma}{Lemma}[section]
\newtheorem{corollary}[theorem]{Corollary}
\newtheorem{example}{Example}[section]
\newtheorem{definition}[problem]{Definition}
\newcommand{\BEQA}{\begin{eqnarray}}
\newcommand{\EEQA}{\end{eqnarray}}
\newcommand{\define}{\stackrel{\triangle}{=}}
\theoremstyle{remark}                              
\newtheorem{rem}{Remark}                           
\begin{document}                                   
\begin{enumerate}[start = 14]                      
\bibliographystyle{IEEEtran}
\vspace{3cm}                                       
\title{Assignment-1 \\Chapter-2\\Complex Number}
\author{AI24BTECH11029- Rudrax Garwa}              
\maketitle
\newpage                                           
\bigskip
\section*{Section-B   JEE MAIN / AIEEE}
\item If $z^2+z+1=0$ ,where z is an imaginary number, then the value of
\begin{`align}
\item $\brak{z+\frac{1}{z}}^2 + \brak{z^2+\frac{1}{z^2}}^2 + \brak{z^3 +\frac{1}{z^3}}^2 +....+\brak{z^6 +\frac{1}{z^6}}^2$
\end{align} 
is

\hfill{[2006]}
\begin{enumerate}
\begin{multicols}{4}
\item 18
\item 54
\item 6
\item 12
\end{multicols}
\end{enumerate}
\item If $|z+4|\leq 3$ , then the maximum value of $|z+1|$ is
\hfill{[2007]}                                     
\begin{enumerate}                                  
\begin{multicols}{4}                               
\item 6                                            
\item 0
\item 4                                            
\item 10
\end{multicols}                                    
\end{enumerate}
\item The conjugate of a complex is $\frac{1}{i-1}$ then that complex number is
\hfill{[2008]}
\begin{multicols}{4}
\begin{enumerate}
\item $\frac{-1}{i-1}$
\item $\frac{1}{i+1}$
\item $\frac{-1}{i+1}$
\item $\frac{1}{i-1}$
\end{enumerate}
\end{multicols}
\item Let R be the real line.Consider the following
subset \\s of the real plane: \\S=\{(x,y):y=x+1 and 0{\textless}x{\textless}2\} \\ T=\{(x,y):x-y is an integer\}, \\Which one of the following is true ?
\hfill{[2008]}
\begin{enumerate}
\begin{multicols}{1}
\item Neither S nor T is an equivalence relation on R                                                  
\columnbreak                                       
\end{multicols}                                    
\begin{multicols}{1}
\item Both S and T are equivalence relation on R   
\columnbreak
\end{multicols}                                    
\begin{multicols}{1}
\item S is an equivalence relation on R but T is not
\columnbreak
\end{multicols}
\begin{multicols}{1}
\item T is an equivalence relation on R but S is not
\columnbreak
\end{multicols}
\end{enumerate}

\item The number of complex numbers z such that $|z-1|=|z+1|=|z-i|$ equals
\hfill{[2010]}

\begin{enumerate}
\begin{multicols}{4}
\item 1
\item 2
\item $\infty$
\item 0
\end{multicols}                                    
\end{enumerate}

\item Let $\alpha,\beta$ be real and z be a complex number. If $z^2 +z\alpha +\beta =0$  has two distinct roots on the line $Rez=1$,then it is necessary that:
\hfill{[2011]}
\begin{enumerate}
\begin{multicols}{2}
\item $\beta \in (-1,0)$
\columnbreak
\item $|\beta|=1$
\end{multicols}
\begin{multicols}{2}
\item $\beta\in (1,\infty)$
\columnbreak                                       
\item $\beta\in (0,1)$
\end{multicols}
\end{enumerate}

\item If $\omega(\neq1)$ is the cube root of unity, and $\brak{1+\omega }^7= A+ B\omega$ .Then $(A,B)$ equals :
\hfill{[2011]}                                     

\begin{enumerate}                                  
\begin{multicols}{2}                               
\item $(1,1)$
\columnbreak                                       
\item $(1,0)$                                      
\end{multicols}                                    
\begin{multicols}{2}
\item $(-1,1)$
\item $(0,1)$
\end{multicols}
\end{enumerate}

\item If $z \neq 1 and \frac{z^2}{z-1}$ is real,thenthe point represented by the complex number z lies:
\hfill{[2012]}

\begin{enumerate}
\begin{multicols}{1}
\item either on the real axis or on a circle not passing through the origin
\end{multicols}                                    
\begin{multicols}{1}                               
\item on a circle with centre at the origin
\end{multicols}
\begin{multicols}{1}                               
\item either on the real axis or on a circle not passing through the origin
\end{multicols}                                    
\begin{multicols}{1}                               
\item on imaginary axis
\end{multicols}                                    
\end{enumerate}

\item If z is a complex number of unit modulus and arguement $\theta$,then the $arg\left(\frac{1+z}{1+\overline{z}}\right)$ equals:
\hfill{[JEE M 2013]}

\begin{enumerate}
\begin{multicols}{4}
\item $-\theta$
\item $\frac{\pi}{2}-\theta$                       
\item $\theta$
\item $\pi-\theta$
\end{multicols}
\end{enumerate}

\item If z is a complex number such that $|z|\geq 2$,then the minimum value of $|z+\frac{1}{2}|$ :     
\hfill{[JEE M 2014]}  

\begin{enumerate}
\begin{multicols}{1}                               
\item is strictly greater than $\frac{5}{2}$       
\end{multicols}                                    
\begin{multicols}{1}
\item is strictly greater than $\frac{3}{2}$ but less than $\frac{5}{2}$
\end{multicols}
\begin{multicols}{1}
\item is equal to $\frac{5}{2}$
\end{multicols}
\begin{multicols}{1}
\item lies in the interval $(1,2)$
\end{multicols}
\end{enumerate}

\item A complex number z is said to be unimodular if$|z|=1$. Suppose $z_1 and z_2$ are complex numbers such that $\frac{z_1 -2z_2}{2-z_1\overline{z_2}}$ is unimodular and $z_2$ is not unimodular. Then the point $z_1$ lies on a:                                
\hfill{[JEE M 2016]}                               

\begin{enumerate}                                  
\begin{multicols}{1}                               
\item circle of radius 2
\end{multicols}                                    
\begin{multicols}{1}                               
\item circle of radius $\sqrt2$
\end{multicols}                                    
\begin{multicols}{1}
\item straight line parallel to x-axis
\end{multicols}
\begin{multicols}{1}
\item straight line parallel to y-axis
\end{multicols}
\end{enumerate}

\item A value of $\theta$ for which $\frac{2+3isin\theta}{1-2isin\theta}$ is purely imaginary is :      
\hfill{[JEE M2016]}

\begin{enumerate}
\begin{multicols}{2}
\item $sin^{-1}\left(\frac{\surd3}{4}\right)$
\columnbreak
\item $sin^{-1}\left(\frac{1}{\surd3}\right)$      
\end{multicols}                                    
\begin{multicols}{2}                                
\item $\frac{\pi}{3}$                               
\columnbreak
\item $\frac{\pi}{6}$                               
\end{multicols}                                     
\end{enumerate}                                     

\item Let  \cbrak{$A=\theta \in \brak{\frac{\pi}{2},\pi}:\frac{3+2isin\theta}{1-2isin\theta}$is purely imaginary.} Then the sum of element in A is:
\hfill{[JEE M 2019-9 Jan (M)]}
\begin{enumerate}
\begin{multicols}{4}
\item $\frac{5\pi}{6}$
\item $\pi$
\item $\frac{3\pi}{4}$
\item $\frac{2\pi}{3}$
\end{multicols}
\end{enumerate}

\item let $\alpha$ and $\beta$ be two roots of the equation $x^2+2x+2=0$,then $\alpha^{15}+\beta^{15}$ is equal to :                                       
\hfill{[JEE M 2019-Jan (M)]}                       
\begin{enumerate}                                   
\begin{multicols}{4}                                
\item $-256$                                        
\item $512$
\item $-512$                                       
\item $256$
\end{multicols}
\end{enumerate}

\item  All the points in the set $S=\brak{\{\frac{\alpha+i}{\alpha-i}:\alpha \in R\}}$ $(i=\sqrt{-1})$ lie on a:
\hfill{[JEE M 2019-9 April (M)]}                    
\begin{enumerate}                                   
\begin{multicols}{1}                                
\item straight line whose slope is 1                
\end{multicols}
\begin{multicols}{1}                                
\item circle whose radius is 1                      
\end{multicols}                                     
\begin{multicols}{1}
\item circle whose radius is $\sqrt2$
\end{multicols}
\begin{multicols}{1}
\item straight line whose slope is -1
\end{multicols}
\end{enumerate}
\end{enumerate}
\end{document}
